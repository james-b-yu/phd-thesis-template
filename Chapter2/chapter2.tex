%!TEX root = ../thesis.tex
%*******************************************************************************
%****************************** Second Chapter *********************************
%*******************************************************************************

\chapter{Background and Preliminaries}\label{cha:2}

\ifpdf
\graphicspath{{Chapter2/Figs/Raster/}{Chapter2/Figs/PDF/}{Chapter2/Figs/}}
\else
\graphicspath{{Chapter2/Figs/Vector/}{Chapter2/Figs/}}
\fi

In this chapter, we establish the conceptual and mathematical preliminaries to lay the necessary groundwork to formally generalise stochastic interpolants \citep{albergo2023stochasticinterpolantsunifyingframework} to function spaces. To achieve this, we structure our discussion as follows.

\begin{enumerate}
  \item We begin by presenting diffusion models (DMs; \citealp{song2021scorebasedgenerativemodelingstochastic,hyvarinen2005estimation,ho2020denoisingdiffusionprobabilisticmodels}) in finite dimensions.
  \item Then, we describe key advantages of the stochastic interpolants framework over DMs, and present a form of stochastic interpolants in their original finite-dimensional context, as proposed by \citet{albergo2023stochasticinterpolantsunifyingframework}.
  \item We define Hilbert spaces as the underlying setting for our analysis, and present an overview of the key mathematical concepts necessary to describe random variables and stochastic differential equations (SDEs) in Hilbert spaces. Given these concepts, we outline key challenges in extending stochastic interpolants to infinite dimensions in Hilbert spaces.
  \item Finally, we provide a review of related works which generalise score-based diffusion models \citep{song2021scorebasedgenerativemodelingstochastic} to function spaces, highlighting the relationship of these methods with their finite-dimensional counterparts.
\end{enumerate}

\section{Diffusion Models in Finite Dimensions}
Diffusion models (DMs; \citealp{song2021scorebasedgenerativemodelingstochastic,hyvarinen2005estimation,ho2020denoisingdiffusionprobabilisticmodels}) are a family of generative models achieving remarkable empirical success across a broad range of domains. To generate data \(x\) distributed according to a target measure \(\mu_{\text{target}}\) on \(N\)-dimensional Euclidean space \(\mathbb{R}^{N}\), we define two stochastic processes on a finite time interval \([0, T]\). For a drift coefficient \(f : [0, T] \times \mathbb{R}^{N} \to \mathbb{R}^{N}\) and diffusion coefficient \(g : [0, T] \times \mathbb{R}^{N} \to \mathbb{R}_{> 0}\), the  \textit{diffusion process} \(X_{t}\), is the solution to the following \textit{forward SDE}:
\[
  \dd{X_{t}} = f(t, X_{t}) \dd{t} + g(t, X_{t}) \dd{W_{t}},\quad X_{0} \sim \mu_{\text{target}},
\]
where \(W_{t}\) is a standard \(N\)-dimensional Wiener process on \([0, T]\).

Let \(\mu_{t}\) be the law (marginal distribution) of \(X_{t}\) and let \(p_{t}: \mathbb{R}^{N} \to \mathbb{R}_{\geq 0}\) be the density of \(\mu_{t}\) with respect to the Lebesgue measure. Under some mild regularity conditions \citep{anderson1982reverse} we may define a \textit{time-reversed process} \(\overline{X}_{t}\), which when solved backwards in time from \(\overline{X}_{T} \sim \mu_{T}\) yields a sample \(\overline{X}_{0} \sim \mu_{\text{target}}\):
\begin{equation}
  \dd{ \overline{X}_{t}} = \qty(f(t, \overline{X}_{t}) - g^{2}(t) \grad \log p_{t}(\overline{X}_{t}))\dd{t} + g(t) \dd{ \overline{W}_{t}},  \quad \overline{X}_{T} \sim \mu_{T}, \label{eqn:dm-reverse}
\end{equation}
where \( \overline{W}_{t}\) is a standard Wiener process when time flows backwards from \(t = T\) to \(0\), and \(\grad \log p_{t}(x)\) is the \textit{score} of the marginal distribution at time \(t\), namely, the spatial derivative of the log-density of \(X_{t}\).

By learning a time-dependent score network \(s_{\theta}(t, x)\) and plugging this in place of \(\grad \log p_{t}(x)\) in \Cref{eqn:dm-reverse}, we may generate approximate samples from \(\mu_{\text{target}}\), provided we have samples from \(\mu_{T}\).

To ensure that \(\mu_{T}\) is a simple and tractable distribution, \(f\) and \(g\) are typically chosen such that the forward process systematically transforms data \(X_{0} \sim \mu_{\text{target}}\) into a Gaussian \(\operatorname{N}(0, \sigma^{2}_{T} I_{N})\). However, this transformation is only guaranteed to be perfect asymptotically as \(T \to \infty\). In a practical implementation, we must terminate time at a finite time step \(T\). This introduces a bias during sampling, since the final condition for the time-reversed SDE is not a Gaussian at time \(T\).

For example, \textit{score-based diffusion models} (SBDMs; \citealp[Equation 11]{song2021scorebasedgenerativemodelingstochastic}) are a special case of DMs in which the forward SDE is a \textit{variance-preserving} SDE which defines a Ornstein-Uhlenbeck process \citep{uhlenbeck1930theory} for a diffusion rate \(b : [0, T] \to \mathbb{R}_{>0}\):
\[
  \dd{X_{t}} = - \frac{1}{2} b(t) X_{t} \dd{t} + \sqrt{b(t)} \dd{W_{t}}, \quad X_{0} \sim \mu_{\text{target}}.
\]
In this case, the law of \(X_{t}\) converges to a standard Gaussian \(\operatorname{N}(0, I_{N})\) in the limit \(t \to \infty\).

While a larger \(T\) bridges the data closer to a Gaussian, a smaller \(T\) helps improve the learned approximation \(s_{\theta}(t, x)\) of the score and leads to more tractable sampling when solving the reverse process. Hence, a tradeoff must be found when choosing \(T\) (see, for example, \citealp{franzese2023much}).

\section{Stochastic Interpolants in Finite Dimensions}
Stochastic interpolants (SIs) are a class of generative models which provide the following improvements in flexibility over DMs:
\begin{enumerate}
  \item SIs can bridge between any two arbitrary distributions determined \textit{a priori}, as opposed to between a single target distribution and a fixed noise prior. Moreover, the source and target distributions can be coupled, allowing SIs to model a joint probability law between source and target data. This provides a powerful and flexible framework, where a single trained model can perform unconditional generation in addition to solving both forward and inverse tasks within a Bayesian setting.
  \item The interpolation is constructed on a finite time horizon, in contrast to DMs which rely on an asymptotic convergence to the simple noise prior.  By design, this has two advantages: it removes approximation bias from the terminal distribution and eliminates the need to tune the time horizon as a hyperparameter. While the \textit{rate} of convergence for the forward process in DMs is exponential for variance-preserving SDEs, a large \(T\) is still required in practice: this makes score estimation more challenging and imposes higher costs for training and sampling \citep{franzese2023much}.
  \item The interpolation path is an explicit design choice, allowing us to construct simple bridges (e.g., linear trajectories) between the two distributions. Simple, low-curvature paths are easier for numerical solvers to approximate accurately, which can lead to greater sampling efficiency with fewer function evaluations. While the path in DMs is determined by the chosen drift and diffusion schedules, recent work \citep{karras2022elucidating,williams2024score} has shown that the most effective sampling trajectories are an important design choice with significant effects on the quality of generated samples. Stochastic interpolants codifies this principle, by making the path itself a primary object of design, rather than having it emerge from a specific SDE formulation.
\end{enumerate}

Each of these merits is demonstrated in a function generation setting in \Cref{cha:4}: we show that our framework is highly effective for solving PDE-based forward and inverse problems. Notably, this is achieved on a strict finite time interval, and with fewer function evaluations and reduced inference time.

Having stated the key merits of SIs over DMs, we now introduce SIs in their finite-dimensional setting, as proposed by \citet{albergo2023stochasticinterpolantsunifyingframework,albergo2023stochastic}. To establish the necessary context for our subsequent development in infinite dimensions, the following discussion captures the conceptual essence of SIs in finite dimensions. A formal and detailed presentation of the specific regularity conditions in our infinite-dimensional setting will be provided in \Cref{cha:3}.

Let \(\mu\) be a joint measure on \(\mathbb{R}^{N} \times \mathbb{R}^{N}\) with marginals \(\mu_{0}\) and \(\mu_{1}\). We draw a (possibly coupled) pair of random variables \(\xi = (\xi_{0}, \xi_{1}) \sim \mu\), where we refer to \(\mu_{0}\) as the \textit{source} and \(\mu_{1}\) as the \textit{target distribution}.

% Furthermore, we denote by \(\mu_{1 \mid 0}(\xi_{0})\) the law of \(\xi_{1}\), conditional on \(\xi_{0}\) and similarly for \(\mu_{0 \mid 1}(\xi_{1})\). We will refer to these, respectively, as the \textit{conditional target} and \textit{conditional source} distributions. The case where the source and target distributions are uncoupled, \(\xi_{0} \perp \xi_{1}\), is a special case of this general setting.

% TODO: might not be necessary to need "source" and "target"

Let \(z\) be a standard \(N\)-dimensional Gaussian, distributed independently of \(\xi\). A \textit{stochastic interpolant} is a family of random variables \(\qty{x_{t}}_{t \in [0, 1]}\) indexed by time \(t \in [0, 1]\):
\[
  x_{t} = \alpha(t) \xi_{0} + \beta(t) \xi_{1} + \gamma(t)z, \quad t \in [0, 1],
\]
where \(\alpha, \beta, \gamma : [0, 1] \to \mathbb{R}_{\geq 0}\) are such that \(\alpha, \beta\) are  continuously differentiable on \([0, 1]\) and \(\gamma\) is continuous on \([0, 1]\) and continuously differentiable on \((0, 1)\). They satisfy the boundary conditions \(\alpha(0) =  \beta(1) = 1\), \(\alpha(1) = \beta(0) = 0\), \(\gamma(0) = \gamma(1) = 0\) and \(\gamma(t) > 0\) for all \(t \in (0, 1)\). We denote their time derivatives respectively by \(\dot{\alpha}, \dot{\beta}, \dot{\gamma}\). Additionally, we denote \(\dot{x}_{t} \coloneqq \dot{\alpha}(t)\xi_{0} + \dot{\beta}(t) \xi_{1} + \dot{\gamma}(t) z\)

Intuitively, the boundary conditions on \(I\) and \(\gamma\) ensure that the law of the stochastic interpolant matches the source and target distributions at the endpoints, \(x_{0} \sim \mu_{0}\) and \(x_{1} \sim \mu_{1}\). For intermmediate times \(t \in (0, 1)\), the law of \(x_{t}\) is equal to that of a deterministic path between \(\xi_{0}\) and \(\xi_{1}\), corrupted by scaled Gaussian noise.

To bridge from \(\mu_{0}\) to \(\mu_{1}\), we choose a positive constant \(\varepsilon > 0\) and define a \textit{forward SDE} as follows:
\[
  \dd{X_{t}} = \qty(\mathop{\mathbb{E}}\qty[\dot{x}_{t} \mid x_{t} = X_{t}] + \varepsilon \grad \log p_{t}(X_{t}))\dd{t} + \sqrt{2\varepsilon} \dd{W_{t}}, \quad X_{0} \sim \mu_{0}, t \in [0, 1],
\]
where \(p_{t}\) is the density of the law of the interpolant \(x_{t}\) at time \(t\), with respect to the Lebesgue measure. Under suitable regularity conditions, \citet{albergo2023stochasticinterpolantsunifyingframework} show that the law of \(X_{t}\) at any time \(t \in [0, 1]\) is equal to the law of \(x_{t}\). Hence, by solving the forward SDE, we generate a sample from the target distribution \(\mu_{1}\). % TODO: refine this notation

Similarly, we define a \textit{time-reversed SDE} which, when solved backwards in time starting from \(\overline{X}_{1} \sim \mu_{1}\), gives a sample from the source distribution \(\mu_{0}\):
\[
  \dd{ \overline{X}_{t}} = \qty(\mathop{\mathbb{E}}\qty[\dot{x}_{t} \mid x_{t} = X_{t}] - \varepsilon \grad \log p_{t}(X_{t}))\dd{t} + \sqrt{2\varepsilon} \dd{ \overline{W}_{t}}, \quad \overline{X}_{1} \sim \mu_{1}, t \in [0, 1].
\]

In the special case where \(\varepsilon = 0\), the forward and time-reversed SDEs collapse to a \textit{probability flow ODE}, where the source of stochasticity only comes from the initial/final conditions, in contrast to \(\varepsilon > 0\) where additional noise is injected by the Wiener process. % TODO: elaborate on transport/probability flow. put into gemini

% acknowledege that we focus only on well-posedness

% TODO: make sure you state the result for

% TODO: add more citations for everytinh

% TODO: make sure paragraphs after equations are indented, but "where"s are not indented

\section{Mathematical Preliminaries}
Generalising stochastic interpolants to infinite dimensions requires confronting several theoretical challenges. To understand these challenges and to construct our infinite-dimensional framework in \Cref{cha:3}, we review some fundamental mathematical preliminaries.

% TODO: add this to intro
% Many data types of modern interest, such as images, time series, or solutions to partial differential equations (PDEs), are fundamentally functions. While a vector in
%   is fully described by a finite list of numbers, a function on a continuous domain requires an infinite number of values to be specified. Practical applications often work with discretizations of these objects—for instance, an image as a grid of pixels—but these are merely finite-dimensional approximations of an underlying, intrinsically infinite-dimensional reality. To analyse such objects rigorously, we require a mathematical framework that can handle this dimensionality. The natural and definitive choice for this purpose is the Hilbert space, an infinite-dimensional vector space equipped with an inner product. This inner product endows the space of functions with a powerful geometric structure, allowing us to generalise familiar concepts like distance, angles, and orthogonality from Euclidean space and providing a robust setting for the analysis that follows.
\paragraph{Hilbert Spaces}
A \textit{Hilbert space} \(H\) is a vector space equipped with a scalar-valued inner product \(\ev{f, g}_{H}\), which is \textit{complete} with respect to the norm \(\norm{f}_{H} \coloneqq \sqrt{\ev{f, f}_{H}}\) induced by this inner product, that is, every \(H\)-valued Cauchy sequence converges in \(H\)-norm to an element in \(H\). The choice of a Hilbert space, as opposed to a more general Banach space, is justified by the fact that the inner product provides essential geometric structure, giving rise to the concept of orthogonality.

Throughout, we let \(H\) be an infinite dimensional Hilbert space satisfying the following two properties:
\begin{enumerate}
  \item \(H\) is \textit{real}, meaning that all scalars, including inner products, are real-valued.
  \item \(H\) is \textit{separable}, which has the implication that there exists a \textit{countable} orthonormal basis for \(H\).
\end{enumerate}

We develop our framework by viewing functions as vectors living in \(H\). Hence, we use the terms \textit{vector} and \textit{function} interchangeably.

% To fix ideas, an example of such a Hilbert space is the set of all square-integrable functions defined on the finite interval \([0, 1]\), and equipped with the inner product
% \[
%   \ev{f, g}_{H} = \int_{[0, 1]} f(x)g(x) \dd{x}.
% \]
% One countable orthonormal basis for this Hilbert space is the trigonometric Fourier series \({1} \cup \qty{\sqrt{2} \cos 2\pi n x, \sqrt{2} \sin 2 \pi n x}_{n=1}^{\infty}\).

% \subsection{Gaussian Measures in Hilbert Spaces}

\paragraph{Gaussian Measures}
A random variable \(x\) is distributed according to a \textit{Gaussian measure}  on a real, separable Hilbert space \(H\) if, for all \(f \in H\), the inner product \(\ev{f, x}_{H} \in \mathbb{R}\) is distributed according to a one-dimensional Gaussian. Such a Gaussian measure is completely determined by its mean \(m \in H\) and a \textit{covariance operator}, defined as a bounded, self-adjoint, positive-semidefinite, linear operator \(C : H \to H\) which satisfies:
\[
  \ev{Cf, g}_{H} = \ev{f, Cg}_{H} = \operatorname{Cov}\qty[\ev{f, x}_{H} \ev{g, x}_{H}] = \mathop{\mathbb{E}}\qty[\ev{f - m, x}_{H} \ev{g - m, x}_{H}],
\]
for all \(f, g \in H\). Hence we denote the law of \(x\) by \(\operatorname{N}(m, C)\).

Let \(\qty{e_{n}}_{n=1}^{\infty}\) be an orthonormal basis of eigenvectors of \(C\) with corresponding eigenvalues \(\qty{\lambda_{n}}_{n=1}^{\infty}\). We call \(C\) \textit{trace class}, if
\[
  \operatorname{Tr}(C) \coloneqq \sum_{n=1}^{\infty} \ev{C e_{n}, e_{n}}_{H} = \sum_{n=1}^{\infty} \lambda_{n} < \infty.
\]

This condition is critical in infinite dimensions: for a Gaussian to be supported on \(H\), its expected squared norm must be finite, and this value is equal to \(\norm{m}^{2}_{H} + \operatorname{Tr}(C)\). A Gaussian with non-trace-class noise will have samples which are almost-surely unbounded in norm and hence do not belong to the Hilbert space \(H\). To ensure that samples are well-defined, we focus only on the case of Gaussians with trace-class covariance.

\paragraph{Cameron-Martin Spaces}

For a covariance operator \(C\), the \textit{Cameron-Martin space}, \(H_{C}\), is an (infinite-dimensional) subspace of \(H\) defined as the image of \(H\) under \(C^{\frac{1}{2}}\). The Cameron-Martin space is a Hilbert space itself when equipped with the inner product \(\ev{f, g}_{H_{C}} \coloneqq \ev{C^{-\frac{1}{2}}f, C^{-\frac{1}{2}}g}_{H}\).

If \(C\) is trace class its eigenvalues must decay to zero. Hence, the eigenvalues of the operator \(C^{-\frac{1}{2}}\) diverge to infinity, making \(C^{-\frac{1}{2}}\) an unbounded operator on \(H\). Critically, this implies that the \(H_{C}\) is a strict, dense subspace of \(H\). An element \(f \in H\) belongs to the subspace \(H_{C}\) only if its coefficients in the eigenbasis of \(C\) decay sufficiently quickly to ensure its Cameron-Martin norm is finite.

Intuitively, since the eigenvalues of \(C\) are typically lowest for high-frequency modes, this condition means that elements of \(H_{C}\) are fundamentally smoother than arbitrary elements of \(H\), as they are constrained to have little energy in their high-frequency components.

If \(H = L^{2}(D, \mu_{D})\) is the set of all square-integrable functions defined on a domain \(D\) with respect to a finite measure \(\mu_{D}\), equipped with the inner product \(\ev{f, g}_{H} = \int_{D} f(x) g(x) \mu_{D}(\dd{x})\), then the Cameron-Martin space \(H_{C}\) for a trace-class covariance operator \(C\) there exists a unique positive-semidefinite kernel function \(k: D \times D \to \mathbb{R}_{\geq 0}\) such that
\[
  Cf(x) = \int_{D} k(x, y)f(y) \mu(\dd{y}), \text{ for all } f \in H.
\]

Consequently, \(H_{C}\) is a reproducing kernel Hilbert space (RKHS) with \(k\) as its reproducing kernel. Intuitively, this provides another reason why \(H_{C}\) is a strict subspace of \(H\): the defining property of the RKHS, that pointwise evaluation of functions is continuous in \(H_{C}\)-norm, imposes a strong regularity condition that functions in \(H_{C}\) are sufficiently smooth.

A fundamental result in the theory of Gaussian measures is that when \(C\) is trace-class, samples from \(N(0, C)\) are almost surely not in \(H_{C}\) even though they belong to the larger space \(H\). % TODO: signpost why this is important + citation

% TODO: give example of gaussian process?

% \paragraph{Densities} In finite dimensions, densities of measures for continuous random variables are typically taken with respect to the Lebesgue measure. However in infinite dimensions, the Lebesgue measure does not exist. Canonically, the reference measure is chosen to be Gaussian. Unlike in finite dimensions, Gaussian measures are in general not equivalen

% TODO: stochastic processes in infinite dimensions

% TODO: wiener process in hilbert space. kourl-louve expansion??

\section{Challenges in Extending SIs to Infinite Dimensions}
Equipped with these mathematical foundations, we now identify the key challenges which arise when extending SIs to infinite dimensions.

\paragraph{Choice of Gaussian Noise} As discussed, samples from a Gaussian \(\operatorname{N}(0, C)\) on \(H\) almost surely do not belong to \(H\) unless \(C\) is trace class. Crucially, this rules out allowing the noise \(z\) in an interpolant to be isotropic.

To construct a well-defined interpolant, we restrict the noise \(z\) to be drawn from a Gaussian with trace-class covariance. We provide design principles for selecting this covariance to achieve desirable properties in the interpolation path. % TODO: make sure this is done!

\paragraph{No Lebesgue measure} Typically in finite dimensions, densities are taken with respect to the Lebesgue measure. However, the Lebesgue measure does not exist in infinite dimensions. Crucially, this makes the score \(\grad \log p_{t}(x)\) ill-defined. One might consider defining the density \(p_{t}\) of the interpolant \(x_{t}\) with respect to some reference Gaussian measure. However due to the time-varying noise schedule \(\gamma(t)z\), this approach faces a crucial obstacle stemming from the Feldman-Hajek theorem: Gaussian measures whose covariance operators are different scaled versions of the same operator are mutually singular. This implies the law of \(x_{t}\) is not absolutely continuous with respect to any single reference Gaussian for all \(t\). % todo: citation for feldman-hajek

To resolve the issue of the ill-defined score, our work extends a key insight from finite-dimensional stochastic interpolants: \citet[][Theorem 2.8]{albergo2023stochasticinterpolantsunifyingframework} show that the score \(\grad \log p_{t}(x)\) can be computed via the conditional expectation \(\frac{1}{\gamma(t)} \mathop{\mathbb{E}}\qty[z \mid x_{t} = x]\). We show a similar principle is true in infinite dimensions. By defining and computing our score operator via a conditional expectation, we avoid the requirement of a global reference measure.

\paragraph{Well-Posedness of SDEs} In finite dimensions, the convolution of interpolated data with scaled noise \(\gamma(t)z\) has a regularising effect, ensuring the corresponding SDE is well-posed. This guarantee is lost in infinite dimensions, where the regularising effect of Gaussian noise on arbitrary measures is often insufficient. This can result in a drift term that is unbounded and/or non-Lipschitz, violating the conditions ensuring the uniqueness or even existence of solutions.

To address this challenge, we establish a set of sufficient conditions on the source and target measures which ensure the drift remains well-behaved, thus guaranteeing the existence and uniqueness of the solution to the infinite-dimensional SDE.

We acknowledge that the sufficient conditions required by our formulation to guarantee a well-posed SDE are strong and unlikely to be strictly met in practice. Nevertheless, we contend that the value of our theoretical framework lies in the design principles it provides for constructing models in empirical settings to ensure stable and well-behaved interpolants.
% TODO: signpost what guidelines?

\section{Related Works}
% TODO: bridge paragraph
% TODO: real literature reivew

\paragraph{Generalisations of DMs in infinite dimensions}

\paragraph{SIs with coupled data}
% TODO: explicitly state that we allow for sampling from conditional distribution; and signpost this in the introduction/contributions area
\paragraph{Forward and inverse problems}

\paragraph{PDE-based forward and inverse problems}

\paragraph{Neural operators}

% TODO: talk about probability flow ODEs (although this may not need to be mentioned specifically in therelated works secition, just signposted later?)

% TODO: define unique weak solution = uniqueness in law

\section{Summary}