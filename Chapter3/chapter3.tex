%!TEX root = ../thesis.tex
%*******************************************************************************
%****************************** Third Chapter **********************************
%*******************************************************************************
\chapter{Construction and Well-Posedness}\label{cha:3}

% **************************** Define Graphics Path **************************
\ifpdf
\graphicspath{{Chapter3/Figs/Raster/}{Chapter3/Figs/PDF/}{Chapter3/Figs/}}
\else
\graphicspath{{Chapter3/Figs/Vector/}{Chapter3/Figs/}}
\fi

In \Cref{cha:2}, we introduced stochastic interpolants (SIs) in their original finite-dimensional setting, noting their advantages over diffusion models (DMs). % todo: elaborate
While DMs have been successfully generalised to achieve state-of-the-art results in function spaces, SIs have not yet been framed in function spaces. Furthermore, existing SI formulations are primarily generative; they do not explicitly guarantee that evolving a process from a point yields a sample from the true conditional target distribution. This conditional sampling capability is essential for the Bayesian inverse problems that are a central motivation for this thesis.

This chapter addresses both of these gaps. We develop a framework for stochastic interpolants on infinite-dimensional Hilbert spaces, explicitly addressing the cases of non-conditional and conditional conditional sampling.

For clarity of presentation, our formal analysis will focus on the process that evolves from the source to the target distribution. The corresponding results for the time-reversed evolution are analogous, and we detail this symmetry in \Cref{cha:3.backwards}. % TODO: signpost throughout this bridge

% TODO: add detail; signpost better
% maybe give summery of the chapter here, eg. we begin

\section{Framework}

We begin by defining the framework of stochastic interpolants in infinite-dimensional Hilbert space.

\begin{definition}[Stochastic interpolant (SI)]
  Let \(H\) be a real, separable Hilbert space equipped with the inner product \(\ev{\cdot, \cdot}_{H}\).
\end{definition}