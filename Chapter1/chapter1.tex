%!TEX root = ../thesis.tex
%*******************************************************************************
%*********************************** First Chapter *****************************
%*******************************************************************************

\chapter{Introduction} \label{cha:1}  %Title of the First Chapter

\ifpdf
\graphicspath{{Chapter1/Figs/Raster/}{Chapter1/Figs/PDF/}{Chapter1/Figs/}}
\else
\graphicspath{{Chapter1/Figs/Vector/}{Chapter1/Figs/}}
\fi

\section{Motivation and Overview}
TODO!

\section{Contributions}
This thesis develops a novel framework for generative modelling on function spaces. Our primary contributions are as follows.
\begin{enumerate}
  \item We formulate stochastic interpolants directly in infinite-dimensional settings, which forms the core of our proposed framework.
  \item We provide a rigourous theoretical analysis, establishing sufficient conditions under which the framework is well-posed and satisfies critical theoretical guarantees.
  \item We translate these theoretical insights into practical design principles to improve the algorithm's performance.
  \item We demonstrate our framework's effectiveness for solving partial differential equation (PDE)-based forward and inverse problems, achieving results competitive with state-of-the-art approaches but with reduced inference time.
  \item Finally, we outline areas for further research, such as extending our theoretical guarantees under more relaxed assumptions  and developing novel practical designs.
\end{enumerate}

% TODO: also say that another specific contribution is conditional sampling

\section{Outline}
This thesis is structured as follows.

\textbf{\Cref{cha:1}} provides the motivation and overview for this thesis.

\textbf{\Cref{cha:2}} presents the necessary groundwork for this thesis: we provide an overview of stochastic interpolants in their original finite-dimensional setting, as proposed by \citet{albergo2023stochasticinterpolantsunifyingframework}, and contrast this with diffusion models for generative modelling \citep{song2021scorebasedgenerativemodelingstochastic}. We then give an overview of the key mathematical concepts necessary to generalise stochastic interpolants to infinite-dimensional spaces, and provide a review of related works in generative modelling on function spaces.

\textbf{\Cref{cha:3}} introduces our core framework: a formulation of stochastic interpolants directly in infinite dimensions. We present a Hilbert space-valued SDE and justify its suitability for generative modelling and prove sufficient conditions for the well-posedness of such an SDE. We provide a training objective and relate this to an error bound of the learned generative process. From this theoretical analysis, we describe how our framework is useful for solving both forward and inverse problems and identify key design principles informing the implemention of our method.

\textbf{\Cref{cha:4}} details an application of our framework for solving PDE-based forward and inverse problems. We describe the datasets and methods used, and compare our results with current state-of-the-art stochastic and deterministic solvers. % TODO: remove this if you do not have it actually.

\textbf{\Cref{cha:5}} describes the merits of our work, as well as some limitations and potential areas for further work.

TODO: mention optimal transport in future work

TODO: make sure you frame the entire paper from the pov of bayesian inverse problems

TODO: add detail!

% TODO: be more specific/refer to results

% TODO: directions for development
% theory. stronger theoretical answers/relax the assumptions
% e.g. relax 1/gamma(t); lipschitz in H-norm, denseness of test functions for theorem 1
% practice. entropic time-change

% Draft:
% 1. formulates stochastic interpolants for the first time in infinite dimensions
% 2. provides sufficient conditions under which this formulation is well-posed and satisfies thereoretical guarantees.
% includes showing that generative SDE indeed results in a sample arbitrarily close to true target distribution (theorem 1)
% proves well-posedness of learned learne

% TODO: nomenclature
% TODO: add and use C1[..; ..] notation

\nomenclature[z-DEM]{DEM}{Discrete Element Method}
\nomenclature[z-FEM]{FEM}{Finite Element Method}
\nomenclature[z-PFEM]{PFEM}{Particle Finite Element Method}
\nomenclature[z-FVM]{FVM}{Finite Volume Method}
\nomenclature[z-BEM]{BEM}{Boundary Element Method}
\nomenclature[z-MPM]{MPM}{Material Point Method}
\nomenclature[z-LBM]{LBM}{Lattice Boltzmann Method}
\nomenclature[z-MRT]{MRT}{Multi-Relaxation
Time}
\nomenclature[z-RVE]{RVE}{Representative Elemental Volume}
\nomenclature[z-GPU]{GPU}{Graphics Processing Unit}
\nomenclature[z-SH]{SH}{Savage Hutter}
\nomenclature[z-CFD]{CFD}{Computational Fluid Dynamics}
\nomenclature[z-LES]{LES}{Large Eddy Simulation}
\nomenclature[z-FLOP]{FLOP}{Floating Point Operations}
\nomenclature[z-ALU]{ALU}{Arithmetic Logic Unit}
\nomenclature[z-FPU]{FPU}{Floating Point Unit}
\nomenclature[z-SM]{SM}{Streaming Multiprocessors}
\nomenclature[z-PCI]{PCI}{Peripheral Component Interconnect}
\nomenclature[z-CK]{CK}{Carman - Kozeny}
\nomenclature[z-CD]{CD}{Contact Dynamics}
\nomenclature[z-DNS]{DNS}{Direct Numerical Simulation}
\nomenclature[z-EFG]{EFG}{Element-Free Galerkin}
\nomenclature[z-PIC]{PIC}{Particle-in-cell}
\nomenclature[z-USF]{USF}{Update Stress First}
\nomenclature[z-USL]{USL}{Update Stress Last}
\nomenclature[s-crit]{crit}{Critical state}
\nomenclature[z-DKT]{DKT}{Draft Kiss Tumble}
\nomenclature[z-PPC]{PPC}{Particles per cell}